% This is samplepaper.tex, a sample chapter demonstrating the
% LLNCS macro package for Springer Computer Science proceedings;
% Version 2.20 of 2017/10/04
%
\documentclass[runningheads]{llncs}
%
\usepackage{graphicx}
\usepackage[utf8]{inputenc}
\usepackage[slovak]{babel}
% Used for displaying a sample figure. If possible, figure files should
% be included in EPS format.
%
% If you use the hyperref package, please uncomment the following line
% to display URLs in blue roman font according to Springer's eBook style:
% \renewcommand\UrlFont{\color{blue}\rmfamily}

\begin{document}
%
\maketitle
\title{Využitie genetického algoritmu na výber funkcií v oblasti behaviorálnej biometrie}
%
%\titlerunning{Abbreviated paper title}
% If the paper title is too long for the running head, you can set
% an abbreviated paper title here
%
\author{Jozef Varga\inst{1}}
%
\authorrunning{F. Author et al.}
% First names are abbreviated in the running head.
% If there are more than two authors, 'et al.' is used.
%
\institute{Fakulta informatiky a informačných technológií STU v Bratislave, 
Ilkovičova 2, 842 16 Bratislava 4}
%
\maketitle              % typeset the header of the contribution
%
\begin{abstract}
    Výber funkcií je v strojovom učení náročnou kombinatorickou úlohou. 
    Funkcie, ktoré neskôr vstupujú do rôznych modelov v strojovom učení, majú
    veľký vplyv na následnú presnosť a rýchlosť samotného modelu. Toto je jeden z problémov,
    ktorý sa vyskytuje pri autorizovaní užívateľa pomocou behavioralnej biomerie. Práve pri
    zaznamenávaní dát o užívateľovy, získame veľké množstvo atributov, ktoré znižujú rýchlosť výpočtu
    a taktiež degradujú kvalitu klasifikácie. 
    
    V tomto článku sme vytvorily genetický algoritmus, ktorý používame práve na výber funkcií. Nami vytvorenú
    metódu porovnávame s bežnými metódami ako sú Variance Inflation Factor (VIF) a vyber funkcií pomocou korelácií. 
    Zvolené metódy boli porovnané pomocou rôznych klasifikátorov ako sú napríklad KNN, SVM, Naive Bayes a iné. 
    V práci sme využili verejný dataset ktorý obsahuje biometrické dáta o užívateľoch. Tieto dáta obsahujú informácie o tom v akom stave sa nachádza užívateľ, teda či sedi: leží, sedí, kráča,  kráča hore schodmy, kráča dole schodmy alebo stojí.
    V tejto práci sa ukázalo, že použitie genetického algoritmu zlepšilo výsledky jednotlivých klasifikátorov.

\keywords{Behavioralna biometria \and Genetický algoritmus \and Strojové učenie \and Výber funkcií / atributov.}
\end{abstract}
%
%
%





%Úvod / Introduction - aká úloha sa ide riešiť, prečo je dobré ju riešiť
\section{Úvod}

%Podobné práce / Related Work - ako riešili vašu alebo podobnú úlohu vexistujúcich prácach - nezabudnite v texte uvádzať citácie na práce.
\section{Podobné práce}

%opis použitého algoritmu - aj s odôvodnením výberu algoritmu
\section{Využité algoritmy}
\subsection{Variance Inflation Factor}
\subsection{Výber funkcií na pomocou korelácií}
\subsection{Genetický algoritmus}

%opis vykonaných experimentov - opis použitého datasetu, opis cieľu a postupuexperimentu, opis nastavení algoritmu (odôvodnenie prečo zvolené nastavenia), opis metrík, ktorými experiment vyhodnocujete... ak sa porovnávate s existujúcimi riešeniami, nezabudnite ich citovať, prípadne aj stručne opísať
\section{Experimenty}

%Vyhodnotenie / Evaluation - prehľadné výsledky experimentov (formou textového opisu, grafov, tabuliek, ...)
\section{Vyhodnotenie}

%Záver / Conclusion
\section{Záver}

%Použitá literatúra / References - používajte LNCS štýl odkazovania sa nazdroje, pokyny nájdete aj v šablóne
\section{Použitá literatúra}















% Please note that the first paragraph of a section or subsection is
% not indented. The first paragraph that follows a table, figure,
% equation etc. does not need an indent, either.

% Subsequent paragraphs, however, are indented.

% \subsubsection{Sample Heading (Third Level)} Only two levels of
% headings should be numbered. Lower level headings remain unnumbered;
% they are formatted as run-in headings.

% \paragraph{Sample Heading (Fourth Level)}
% The contribution should contain no more than four levels of
% headings. Table~\ref{tab1} gives a summary of all heading levels.

% \begin{table}
% \caption{Table captions should be placed above the
% tables.}\label{tab1}
% \begin{tabular}{|l|l|l|}      
% \hline
% Heading level &  Example & Font size and style\\
% \hline
% Title (centered) &  {\Large\bfseries Lecture Notes} & 14 point, bold\\
% 1st-level heading &  {\large\bfseries 1 Introduction} & 12 point, bold\\
% 2nd-level heading & {\bfseries 2.1 Printing Area} & 10 point, bold\\
% 3rd-level heading & {\bfseries Run-in Heading in Bold.} Text follows & 10 point, bold\\
% 4th-level heading & {\itshape Lowest Level Heading.} Text follows & 10 point, italic\\
% \hline
% \end{tabular}
% \end{table}


% \noindent Displayed equations are centered and set on a separate
% line.
% \begin{equation}
% x + y = z
% \end{equation}
% Please try to avoid rasterized images for line-art diagrams and
% schemas. Whenever possible, use vector graphics instead (see
% Fig.~\ref{fig1}).

% \begin{figure}
% \includegraphics[width=\textwidth]{image/fig1.png}
% \caption{A figure caption is always placed below the illustration.
% Please note that short captions are centered, while long ones are
% justified by the macro package automatically.} \label{fig1}
% \end{figure}

% \begin{theorem}
% This is a sample theorem. The run-in heading is set in bold, while
% the following text appears in italics. Definitions, lemmas,
% propositions, and corollaries are styled the same way.
% \end{theorem}
%
% the environments 'definition', 'lemma', 'proposition', 'corollary',
% 'remark', and 'example' are defined in the LLNCS documentclass as well.
%


% \begin{proof}
% Proofs, examples, and remarks have the initial word in italics,
% while the following text appears in normal font.
% \end{proof}
% For citations of references, we prefer the use of square brackets
% and consecutive numbers. Citations using labels or the author/year
% convention are also acceptable. The following bibliography provides
% a sample reference list with entries for journal
% articles~\cite{ref_article1}, an LNCS chapter~\cite{ref_lncs1}, a
% book~\cite{ref_book1}, proceedings without editors~\cite{ref_proc1},
% and a homepage~\cite{ref_url1}. Multiple citations are grouped
% \cite{ref_article1,ref_lncs1,ref_book1},
% \cite{ref_article1,ref_book1,ref_proc1,ref_url1}.


%
% ---- Bibliography ----
%
% BibTeX users should specify bibliography style 'splncs04'.
% References will then be sorted and formatted in the correct style.
%
% \bibliographystyle{splncs04}
% \bibliography{mybibliography}
%




% \begin{thebibliography}{8}
% \bibitem{ref_article1}
% Author, F.: Article title. Journal \textbf{2}(5), 99--110 (2016)

% \bibitem{ref_lncs1}
% Author, F., Author, S.: Title of a proceedings paper. In: Editor,
% F., Editor, S. (eds.) CONFERENCE 2016, LNCS, vol. 9999, pp. 1--13.
% Springer, Heidelberg (2016). \doi{10.10007/1234567890}

% \bibitem{ref_book1}
% Author, F., Author, S., Author, T.: Book title. 2nd edn. Publisher,
% Location (1999)

% \bibitem{ref_proc1}
% Author, A.-B.: Contribution title. In: 9th International Proceedings
% on Proceedings, pp. 1--2. Publisher, Location (2010)

% \bibitem{ref_url1}
% LNCS Homepage, \url{http://www.springer.com/lncs}. Last accessed 4
% Oct 2017
% \end{thebibliography}
\end{document}
